\documentclass{article}
\usepackage{zed-csp}
\title{A data structure for representing text}
\author{David Faitelson}

\begin{document}

\maketitle

\section{Characters} 

A character can be either a printable character (that we call
a letter) or a separator. 
There are two kinds of separators, a new line separator and a
page separator.

\begin{zed}
  Char ::= A | B | NEWL | NEWP | SPACE 
\end{zed}

\begin{zed}
  Letter == \{ A,B \} \also
  Separator == \{ NEWL, NEWP, SPACE \}
\end{zed}

\section{An abstract model}

The abstract representation of a text editor is that it has a
sequence of characters addressable using an address data type that
represents the index of a character in the sequence, and supporting
the following operations: insert a character after an address,
erasing characters in a range of addresses, and searching forward
and backward for a substring.

\begin{schema}{AMachine}
  text : \seq Char
\end{schema}

\begin{schema}{AAddress}
  index: \nat
\end{schema}

\begin{axdef}
  astart : AMachine \fun AAddress
\where
  \forall m : AMachine @ \\
\t1 (astart~m) . index = 0
\end{axdef} 

\begin{axdef}
  aend : AMachine \fun AAddress
\where
  \forall m : AMachine @ \\
\t1 (aend~m) . index = \# m.text
\end{axdef} 

\begin{axdef}
  precedes : AAddress \rel AAddress 
\where
  \forall b,e : AAddress @ \\
\t1 b \mapsto e \in precedes \iff \\
\t2   b.index \leq e.index
\end{axdef}

\begin{axdef}
  distance : AAddress \cross AAddress \pfun \nat
\where
 \forall x,y : AAddress | x \inrel{precedes} y @ \\
\t1 distance~(x,y) = y.index - x.index
\end{axdef}

The function $atext$ associates each valid address range with the
sequence of characters that this range marks in the text.

\begin{schema}{Atext}
  AMachine \\
  atext : (AAddress \cross AAddress) \pfun \seq Char
\where
  \forall b,e : AAddress | \\
\t1 astart~\theta AMachine \inrel{precedes} b \land {} \\
\t1 b \inrel{precedes} e \land {} \\
\t1 e \inrel{precedes} aend~\theta AMachine @ \\
\t1 \exists u,v,w : \seq Char @ \\
\t2   text = u \cat v \cat w \land {} \\
\t2   distance~(astart~ \theta AMachine, b) = \# u \land {} \\
\t2   distance~(b,e) = \# v \land {} \\
\t2   atext~(b,e) = v
\end{schema}

The function $ainc$ maps an address to the address immediatley
following it. Unless it is pointing one past the end of the text
area. In that case incrementing it takes it back to the beginning.

\begin{schema}{Ainc}
  AMachine \\
  ainc : AAddress \fun AAddress
\where
  \forall addr : AAddress | addr.index < \# text @ \\
\t1 (ainc~addr).index = addr.index + 1 \\
  \forall addr : AAddress | addr.index = \# text @ \\
\t1 (ainc~addr).index = 0
\end{schema}

\begin{schema}{AInit}
  AMachine'
\where
  text' = \emptyset
\end{schema}

The operation $AInsert$ inserts a character after the location
indicated by the address. This poses the quetion of how to insert
a character at the beginning of the text. For that we use the address
that represents one place after the end of the text. In other words
to insert a character at the beginning of the text we pass $aend$
as the address.

\begin{schema}{AInsertAtStart}
  \Delta AMachine \\
  addr? : AAddress \\
  char? : Char
\where
  addr? = aend~\theta AMachine \\
  text' = \langle char? \rangle \cat text
\end{schema}

\begin{schema}{AInsertAfterStart}
  \Delta AMachine \\
  Ainc \\
  Atext \\
  addr? : AAddress \\
  char? : Char
\where
  addr? \inrel{precedes} aend~\theta AMachine \\
  addr?  \neq aend~\theta AMachine \\
  text' = atext~(astart~\theta AMachine, ainc~addr?) \cat \langle char? \rangle \cat atext~(ainc~addr?, aend~\theta AMachine)
\end{schema}

\begin{zed}
  AInsert \defs AInsertAtStart \lor AInsertAfterStart
\end{zed}


\begin{schema}{AErase}
  \Delta AMachine \\
  Atext \\
  b? : AAddress\\
  e? : AAddress 
\where
  b? \inrel{precedes} e? \\
  e? \inrel{precedes} (aend~\theta AMachine) \\
  text' = atext~(astart~\theta AMachine, b?) \cat atext~(e?, aend~\theta AMachine) 
\end{schema}

\section{First refinement}

We define a machine that holds the text in a doubly linked list of
lines. An address is then a pair of line and index in the line.

\begin{schema}{Line}
  text : \seq Char \\
\end{schema}

\begin{schema}{CAddress}
  line : Line \\
  index : \nat
\end{schema}

The line $end$ marks the position one-past-the-end. If the text
area is not empty then $next~end$ is the first line. 

\begin{schema}{CLines}
  next : Line \inj Line \\ 
  prev : Line \inj Line \\ 
  end : Line
\where
  next = prev \inv \\ 
  next\star \limg \{ end \} \rimg = \dom next \\
  end.text = \langle \rangle 
\end{schema}

To match the text in the abstract model we concatenate the text in
the lines according to the $next$ relation starting from $next~end$.

\begin{schema}{CollectText}
  CLines \\
  fragments : \seq (\seq Char)
\where
  next~end \neq end \\
  \forall i : \dom fragments @ \\
\t1 fragments~i = (next^{i+1} end).text
\end{schema}

Note that in an empty state the only line is $end$ which has no text. 
Thus in that case $\dom fragments = \{0\}$ and $next~end = end$ thus we have 

\begin{argue}
fragments~0 = \\
\t1 (next^{0+1}~ end).text = \\
\t1 (next~ end).text = \\
\t1 end.text = \langle\rangle.
\end{argue}

We use the function $concat$ to concatenate a sequence of sequences
into a single sequence

\begin{gendef}[T]
  concat: \seq (\seq T) \fun \seq T
\where
  concat~\langle \rangle = \langle \rangle \\
  \forall s : \seq T; t : \seq (\seq T) @ \\
\t1 concat~ (\langle s \rangle \cat t) = s \cat (concat~t)
\end{gendef}

\begin{schema}{RetrieveText}
  AMachine \\
  CollectText \\
\where
  text = concat~fragments
\end{schema}

A line $x$ is $before$ line $y$ if it is possible to reach from $x$
to $y$ by following the $next$ relation without hitting $end$ along
the way.

\begin{schema}{LineOrder}
  CLines \\
  before : Line \rel Line
\where
  \forall x,y : Line @ x \inrel{before} y \iff \\
\t1  (\exists n : \nat @ next^{n} ~ x = y \land {} \\
\t2    (\forall k : \nat | k < n @ next^{k} ~ x \neq end))
\end{schema}

In particular this gives for any line $x$

\begin{argue}
  next^{0}~x = x & because $next^{0}$ is the identity relation \\
\t1 \implies next^{0}~x = x  \\
\t1 \implies next^{0}~x = x \land {} \\
\t2            (\forall k: \nat | k < 0 @ next^{k} ~ x \neq end) \\
  & because the universal quantifier ranges over the empty set \\
\t1 \implies x \inrel{before} x 
\end{argue}

All the lines appear before $end$ because 
$next\star \limg end\rimg = \dom next$ 
and as $next$ is a function
the only time when we can have $next~end = end$ is when the text
area is empty and thus has no lines.

The function $sumLength$ maps a finite set of lines to the sum of
the length of their text.

\begin{axdef}
  sumLength : \power_1 Line \fun \nat
\end{axdef}

We now define a retrieve relation to translate between abstract
and concrete addresses

\begin{schema}{RetrieveAddress}
  AAddress[aindex/index] \\
  CAddress \\
  LineOrder 
\where
  aindex = index + sumLength~\{ x : Line | x \inrel{before} line \land x \neq line \} 
\end{schema}

We will now show that the abstract address of the end of the text
area corresponds to the concrete address $(end, 0)$.

\begin{argue}
aend~\theta AMachine.index
= \# m.text \\
= 0 + \# m.text \\
= 0 + sumLength~(\dom next \setminus \{end\}) \\
= 0 + sumLength~\{ x : Line | x \inrel{before} end \land x \neq end \} \\
= index + sumLength~\{ x : Line | x \inrel{before} end \land x \neq end \} 
\end{argue}

The function $cinc$ maps a concrete address to the address immediatley 
following it.

\begin{schema}{CInc}
  CLines \\
  cinc : CAddress \fun CAddress
\where
  \forall addr : CAddress | addr.index < \# addr.line.text - 1 @ \\
\t1 (cinc~addr).line = addr.line \land {} \\
\t1 (cinc~addr).index = addr.index + 1 \\
  \forall addr : CAddress | addr.index = \# addr.line.text - 1 @ \\
\t1  (cinc~addr).line = next~addr.line \land {} \\
\t1  (cinc~addr).index = 0 \\
  \forall addr : CAddress | addr.line = end \land addr.index = 0 @ \\
\t1  (cinc~addr).line = next~end \land {} \\
\t1  (cinc~addr).index = 0
\end{schema}

Inserting a character after an address.

\begin{schema}{CInsertAtStart}
  CLines \\
  addr? : CAddress \\
  char? : Char \\
  \Delta Line \\
\where
  addr?.line = end \\
  addr?.index = 0 \\
  next~end \neq end \\
  next~end = \theta Line \\
  text' = \langle char? \rangle \cat text 
\end{schema}

Note that in general inserting or removing text from lines does not
change the linked list structure.

\begin{schema}{CInsertAtEnd}
  CLines \\
  addr? : CAddress \\
  char? : Char \\
  \Delta Line \\
\where
  addr?.line = end \\
  addr?.index = 0 \\
  next~end \neq end \\
  next~end = \theta Line \\
  text' = \langle char? \rangle \cat text 
\end{schema}

\section{Shifting}

A token is either a single separator or a non empty sequence of non
separator characters.

There are two ways we can move tokens between lines without changing
the abstract representation. We can shift the last token in a line to
the next line, or we can shift the first token in a line to the previous
line. 


\end{document}

